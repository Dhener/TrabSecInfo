\section{Comparação de Implementações: Segurança e Evidências}

A comparação entre as implementações sem e com TLS é fundamental para demonstrar o valor da criptografia em redes. Esta seção contrasta os protocolos com base nas garantias de segurança e utiliza as capturas de pacotes de rede (Wireshark) como evidência empírica.

\subsection{1. Confidencialidade (O Conteúdo dos Dados)}

\subsubsection{Transmissão Sem TLS (TCP Puro)}
Na implementação sem TLS (porta 5001), o protocolo TCP puro não adiciona nenhuma camada de segurança, e a transmissão ocorre em \textbf{texto plano}.
\begin{itemize}
    \item \textbf{Análise:} A captura de pacotes (Figura \ref{fig:tcp_texto_plano}) mostra o conteúdo literal do arquivo de teste ("Este é um arquivo de teste...") visível no \textit{payload} do segmento TCP.
    \item \textbf{Conclusão:} A confidencialidade é \textbf{nula}. Qualquer \textit{sniffing} na rede expõe os dados.
\end{itemize}

\begin{figure}[H]
    \centering
    \includegraphics[width=0.9\textwidth]{./imagens/Screenshot 2025-11-08 at 15.42.45.png} % Imagem da Captura TCP Puro
    \caption{Evidência de Texto Plano (TCP Puro). Conteúdo visível no payload.}
    \label{fig:tcp_texto_plano}
\end{figure}

\subsubsection{Transmissão Com TLS}
Na implementação com TLS (porta 5002), a chave de sessão secreta (negociada no \textit{handshake}) é usada para criptografar o arquivo antes do envio.
\begin{itemize}
    \item \textbf{Análise:} A captura (Figura \ref{fig:tls_cifrado}) mostra que o tráfego de dados é categorizado como \texttt{TLSv1.2 Record Layer: Application Data}. O \textit{payload} contém apenas sequências ilegíveis de bytes cifrados.
    \item \textbf{Conclusão:} A confidencialidade é \textbf{garantida}, pois os dados estão protegidos contra interceptação.
\end{itemize}

\begin{figure}[H]
    \centering
    \includegraphics[width=0.9\textwidth]{./imagens/texto_cifrado.png} % Imagem da Captura TLS Cifrado
    \caption{Evidência de Envio Cifrado (TLS). Payload ilegível (Encrypted Application Data).}
    \label{fig:tls_cifrado}
\end{figure}

\subsection{2. Integridade e Autenticidade}

\begin{table}[H]
    \centering
    \resizebox{1.0\textwidth}{!}{
    \begin{tabular}{|l|c|c|}
    \hline
    \textbf{Garantia de Segurança} & \textbf{TCP Puro (Sem TLS)} & \textbf{TLS} \\
    \hline
    \textbf{Autenticidade (Identidade)} & Nenhuma & \textbf{Garantida} (via Certificado Digital) \\
    \hline
    \textbf{Integridade (Anti-adulteração)} & Nenhuma & \textbf{Garantida} (via MAC/Hash) \\
    \hline
    \end{tabular}}
    \caption{Comparação de Mecanismos de Segurança}
    \label{tab:seguranca_mecanismos}
\end{table}

\begin{itemize}
    \item \textbf{Autenticidade:} Na implementação TCP, não há verificação de identidade; o cliente aceita a conexão de qualquer servidor na porta 5001. No TLS, o servidor envia seu certificado digital para provar sua identidade durante o \textit{handshake}, impedindo ataques \textit{Man-in-the-Middle}.
    \item \textbf{Integridade:} O TCP usa apenas um \textit{checksum} básico. O TLS, por outro lado, anexa um \textbf{Message Authentication Code (MAC)} criptográfico a cada registro de dados, garantindo que o pacote não foi alterado intencionalmente no caminho.
\end{itemize}